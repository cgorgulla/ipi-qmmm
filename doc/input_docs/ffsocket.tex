\section{FFSOCKET}
\label{FFSOCKET}
\begin{ipifield}{}%
{Deals with the assigning of force calculation jobs to different driver codes, and collecting the data, using a socket for the data communication.}%
{}%
{\ipiitem{mode}%
{Specifies whether the driver interface will listen onto a internet socket [inet] or onto a unix socket [unix].}%
{default: `inet'; data type: string; options: `unix', `inet'; }%
\ipiitem{pbc}%
{Applies periodic boundary conditions to the atoms coordinates before passing them on to the driver code.}%
{default:  True ; data type: boolean; }%
\ipiitem{name}%
{Mandatory. The name by which the forcefield will be identified in the System forces section.}%
{data type: string; }%
}
\begin{ipifield}{latency}%
{The number of seconds the polling thread will wait between exhamining the list of requests.}%
{default:  0.01 ; data type: float; }%
{}
\end{ipifield}
\begin{ipifield}{timeout}%
{This gives the number of seconds before assuming a calculation has died. If 0 there is no timeout.}%
{default:  0.0 ; data type: float; }%
{}
\end{ipifield}
\begin{ipifield}{parameters}%
{The parameters of the force field}%
{default:  \{ \} ; data type: dictionary; }%
{}
\end{ipifield}
\begin{ipifield}{address}%
{This gives the server address that the socket will run on.}%
{default: `localhost'; data type: string; }%
{}
\end{ipifield}
\begin{ipifield}{slots}%
{This gives the number of client codes that can queue at any one time.}%
{default:  4 ; data type: integer; }%
{}
\end{ipifield}
\begin{ipifield}{port}%
{This gives the port number that defines the socket.}%
{default:  65535 ; data type: integer; }%
{}
\end{ipifield}
\end{ipifield}
