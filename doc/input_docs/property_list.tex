\ipiitem{atom\_f}{The force (x,y,z) acting on a particle given its index. Takes arguments index
                      and bead (both zero based). If bead is not specified, refers to the centroid.}{dimension: force; size: 3; }\ipiitem{atom\_p}{The momentum (x,y,z) of a particle given its index. Takes arguments index
                      and bead (both zero based). If bead is not specified, refers to the centroid.}{dimension: momentum; size: 3; }\ipiitem{atom\_v}{The velocity (x,y,z) of a particle given its index. Takes arguments index
                       and bead (both zero based). If bead is not specified, refers to the centroid.}{dimension: velocity; size: 3; }\ipiitem{atom\_x}{The position (x,y,z) of a particle given its index. Takes arguments index
                       and bead (both zero based). If bead is not specified, refers to the centroid.}{dimension: length; size: 3; }\ipiitem{cell\_abcABC}{The lengths of the cell vectors and the angles between them in degrees as a list of the
                      form $[$a, b, c, A, B, C$]$, where A is the angle between the sides of length b and c in degrees, and B and C
                      are defined similarly. Since the output mixes different units, a, b and c can only be output in bohr.}{size: 6; }\ipiitem{cell\_h}{The simulation cell as a matrix. Returns the 6 non-zero components in the form $[$xx, yy, zz, xy, xz, yz$]$.}{dimension: length; size: 6; }\ipiitem{chin\_weight}{The 3 numbers output are 1) the logarithm of the weighting factor -beta\_P delta H,
                      2) the square of the logarithm, and 3) the weighting factor}{size: 3; }\ipiitem{conserved}{The value of the conserved energy quantity per bead.}{dimension: energy; }\ipiitem{density}{The mass density of the physical system.}{dimension: density; }\ipiitem{displacedpath}{This is the estimator for the end-to-end distribution, that can be used to calculate the
                      particle momentum distribution as described in in L. Lin, J. A. Morrone, R. Car and M. Parrinello,
                      105, 110602 (2010), Phys. Rev. Lett. Takes arguments 'ux', 'uy' and 'uz', which are the components of
                      the path opening vector. Also takes an argument 'atom', which can be either an atom label or index
                      (zero based) to specify which species to find the end-to-end distribution estimator for. If not
                      specified, all atoms are used. Note that one atom is computed at a time, and that each path opening
                      operation costs as much as a PIMD step. Returns the average over the selected atoms of the estimator of
                      exp(-U(u)) for each frame.}{}\ipiitem{ensemble\_temperature}{The target temperature for the current ensemble}{dimension: temperature; }\ipiitem{forcemod}{The modulus of the force. With the optional argument 'bead'
                         will print the force associated with the specified bead.}{dimension: force; }\ipiitem{isotope\_scfep}{Returns the (many) terms needed to compute the scaled-coordinates free energy
                      perturbation scaled mass KE estimator (M. Ceriotti, T. Markland, J. Chem. Phys. 138, 014112 (2013)).
                      Takes two arguments, 'alpha' and 'atom', which give the
                      scaled mass parameter and the atom of interest respectively, and default to '1.0' and ''. The
                      'atom' argument can either be the label of a particular kind of atom, or an index (zero based)
                      of a specific atom. This property computes, for each atom in the selection, an estimator for
                      the kinetic energy it would have had if it had the mass scaled by alpha. The 7 numbers output
                      are the average over the selected atoms of the log of the weights $<$h$>$, the average of the
                      squares $<$h**2$>$, the average of the un-weighted scaled-coordinates kinetic energies  $<$T\_CV$>$
                      and of the squares $<$T\_CV**2$>$, the log sum of the weights LW=ln(sum(e**(-h))), the sum of the
                      re-weighted kinetic energies, stored as a log modulus and sign, LTW=ln(abs(sum(T\_CV e**(-h))))
                      STW=sign(sum(T\_CV e**(-h))). In practice, the best estimate of the estimator can be computed
                      as $[$sum\_i exp(LTW\_i)*STW\_i$]$/$[$sum\_i exp(LW\_i)$]$. The other terms can be used to compute diagnostics
                      for the statistical accuracy of the re-weighting process. Note that evaluating this estimator costs
                      as much as a PIMD step for each atom in the list. The elements that are output have different
                      units, so the output can be only in atomic units.}{size: 7; }\ipiitem{isotope\_tdfep}{Returns the (many) terms needed to compute the thermodynamic free energy
                      perturbation scaled mass KE estimator (M. Ceriotti, T. Markland, J. Chem. Phys. 138, 014112 (2013)).
                      Takes two arguments, 'alpha' and 'atom', which give the
                      scaled mass parameter and the atom of interest respectively, and default to '1.0' and ''. The
                      'atom' argument can either be the label of a particular kind of atom, or an index (zero based)
                      of a specific atom. This property computes, for each atom in the selection, an estimator for
                      the kinetic energy it would have had if it had the mass scaled by alpha. The 7 numbers output
                      are the average over the selected atoms of the log of the weights $<$h$>$, the average of the
                      squares $<$h**2$>$, the average of the un-weighted scaled-coordinates kinetic energies  $<$T\_CV$>$
                      and of the squares $<$T\_CV**2$>$, the log sum of the weights LW=ln(sum(e**(-h))), the sum of the
                      re-weighted kinetic energies, stored as a log modulus and sign, LTW=ln(abs(sum(T\_CV e**(-h))))
                      STW=sign(sum(T\_CV e**(-h))). In practice, the best estimate of the estimator can be computed
                      as $[$sum\_i exp(LTW\_i)*STW\_i$]$/$[$sum\_i exp(LW\_i)$]$. The other terms can be used to compute diagnostics
                      for the statistical accuracy of the re-weighting process. Evaluating this estimator is inexpensive,
                      but typically the statistical accuracy is worse than with the scaled coordinates estimator.
                      The elements that are output have different
                      units, so the output can be only in atomic units.}{size: 7; }\ipiitem{isotope\_zetasc}{Returns the (many) terms needed to directly compute the relative probablity of
                      isotope substitution in two different systems/phases. Takes four arguments, 'alpha' , which gives the
                      scaled mass parameter and default to '1.0', and 'atom', which is the label or index of a type of atoms.
                      The 3 numbers output are 1) the average over the excess potential energy for scaled coordinates $<$sc$>$,
                      2) the average of the squares of the excess potential energy $<$sc**2$>$, and 3) the average of the exponential
                      of excess potential energy $<$exp(-beta*sc)$>$}{size: 3; }\ipiitem{isotope\_zetasc\_4th}{Returns the (many) terms needed to compute the scaled-coordinates fourth-order direct estimator.
					  Takes two arguments, 'alpha' , which gives the scaled mass parameter and default to '1.0', and 'atom',
					  which is the label or index of a type of atoms.
                      The 5 numbers output are 1) the average over the excess potential energy for an isotope atom substitution $<$sc$>$,
                      2) the average of the squares of the excess potential energy $<$sc**2$>$, and 3) the average of the exponential
                      of excess potential energy $<$exp(-beta*sc)$>$, and 4-5) Suzuki-Chin and Takahashi-Imada 4th-order reweighing term}{size: 5; }\ipiitem{isotope\_zetatd}{Returns the (many) terms needed to directly compute the relative probablity of
                      isotope substitution in two different systems/phases. Takes two arguments, 'alpha' , which gives the
                      scaled mass parameter and default to '1.0', and 'atom', which is the label or index of a type of atoms.
                      The 3 numbers output are 1) the average over the excess spring energy for an isotope atom substitution $<$spr$>$,
                      2) the average of the squares of the excess spring energy $<$spr**2$>$, and 3) the average of the exponential
                      of excess spring energy $<$exp(-beta*spr)$>$}{size: 3; }\ipiitem{isotope\_zetatd\_4th}{Returns the (many) terms needed to compute the thermodynamic fourth-order direct estimator.
					  Takes two arguments, 'alpha' , which gives the scaled mass parameter and default to '1.0', and 'atom',
					  which is the label or index of a type of atoms.
                      The 5 numbers output are 1) the average over the excess spring energy for an isotope atom substitution $<$spr$>$,
                      2) the average of the squares of the excess spring energy $<$spr**2$>$, and 3) the average of the exponential
                      of excess spring energy $<$exp(-beta*spr)$>$, and 4-5) Suzuki-Chin and Takahashi-Imada 4th-order reweighing term}{size: 5; }\ipiitem{kinetic\_cv}{The centroid-virial quantum kinetic energy of the physical system.
                      Takes an argument 'atom', which can be either an atom label or index (zero based)
                      to specify which species to find the kinetic energy of. If not specified, all atoms are used.}{dimension: energy; }\ipiitem{kinetic\_ij}{The centroid-virial off-diagonal quantum kinetic energy tensor of the physical system.
                      This computes the cross terms between atoms i and atom j, whose average is  $<$p\_i*p\_j/(2*sqrt(m\_i*m\_j))$>$.
                      Returns the 6 independent components in the form $[$xx, yy, zz, xy, xz, yz$]$. Takes arguments 'i' and 'j',
                       which give the indices of the two desired atoms.}{dimension: energy; size: 6; }\ipiitem{kinetic\_md}{The kinetic energy of the (extended) classical system.
                       Takes optional arguments 'atom', 'bead' or 'nm'.  'atom' can be either an
                       atom label or an index (zero-based) to specify which species or individual atom
                       to output the kinetic energy of. If not specified, all atoms are used and averaged.
                       'bead' or 'nm' specify whether the kinetic energy should be computed for a single bead
                       or normal mode. If not specified, all atoms/beads/nm are used.}{dimension: energy; }\ipiitem{kinetic\_td}{The primitive quantum kinetic energy of the physical system.
                      Takes an argument 'atom', which can be either an atom label or index (zero based)
                      to specify which species to find the kinetic energy of. If not specified, all atoms are used.}{dimension: energy; }\ipiitem{kinetic\_tens}{The centroid-virial quantum kinetic energy tensor of the physical system.
                      Returns the 6 independent components in the form $[$xx, yy, zz, xy, xz, yz$]$. Takes an
                      argument 'atom', which can be either an atom label or index (zero based) to specify
                      which species to find the kinetic tensor components of. If not specified, all atoms are used.}{dimension: energy; size: 6; }\ipiitem{kstress\_cv}{The quantum estimator for the kinetic stress tensor of the physical system.
                      Returns the 6 independent components in the form $[$xx, yy, zz, xy, xz, yz$]$.}{dimension: pressure; size: 6; }\ipiitem{kstress\_md}{The kinetic stress tensor of the (extended) classical system. Returns the 6
                      independent components in the form $[$xx, yy, zz, xy, xz, yz$]$.}{dimension: pressure; size: 6; }\ipiitem{pot\_component}{The contribution to the system potential from one of the force components. Takes one mandatory
                         argument index (zero-based) that indicates which component of the potential must be returned. The optional argument 'bead'
                         will print the potential associated with the specified bead. }{dimension: energy; }\ipiitem{potential}{The physical system potential energy. With the optional argument 'bead'
                         will print the potential associated with the specified bead.}{dimension: energy; }\ipiitem{pressure\_cv}{The quantum estimator for pressure of the physical system.}{dimension: pressure; }\ipiitem{pressure\_md}{The pressure of the (extended) classical system.}{dimension: pressure; }\ipiitem{r\_gyration}{The average radius of gyration of the selected ring polymers. Takes an
                      argument 'atom', which can be either an atom label or index (zero based) to specify which
                      species to find the radius of gyration of. If not specified, all atoms are used and averaged.}{dimension: length; }\ipiitem{scaledcoords}{Returns the estimators that are required to evaluate the scaled-coordinates estimators
                       for total energy and heat capacity, as described in T. M. Yamamoto,
                       J. Chem. Phys., 104101, 123 (2005). Returns eps\_v and eps\_v', as defined in that paper.
                       As the two estimators have a different dimensions, this can only be output in atomic units.
                       Takes one argument, 'fd\_delta', which gives the value of the finite difference parameter used -
                       which defaults to -1e-05. If the value of 'fd\_delta' is negative,
                       then its magnitude will be reduced automatically by the code if the finite difference error
                       becomes too large.}{size: 2; }\ipiitem{spring}{The total spring potential energy between the beads of all the ring polymers in the system.}{dimension: energy; }\ipiitem{step}{The current simulation time step.}{dimension: number; }\ipiitem{stress\_cv}{The total quantum estimator for the stress tensor of the physical system. Returns the
                      6 independent components in the form $[$xx, yy, zz, xy, xz, yz$]$.}{dimension: pressure; size: 6; }\ipiitem{stress\_md}{The total stress tensor of the (extended) classical system. Returns the 6
                      independent components in the form $[$xx, yy, zz, xy, xz, yz$]$.}{dimension: pressure; size: 6; }\ipiitem{temperature}{The current temperature, as obtained from the MD kinetic energy of the (extended)
                                      ring polymer. Takes optional arguments 'atom', 'bead' or 'nm'.  'atom' can be either an
                                      atom label or an index (zero-based) to specify which species or individual atom
                                      to output the temperature of. If not specified, all atoms are used and averaged.
                                      'bead' or 'nm' specify whether the temperature should be computed for a single bead
                                      or normal mode.}{dimension: temperature; }\ipiitem{ti\_pot}{The correction potential in Takahashi-Imada 4th-order PI expansion.
                             Takes an argument 'atom', which can be either an atom label or index (zero based)
                             to specify which species to find the correction term for. If not specified, all atoms are used.}{dimension: energy; size: 1; }\ipiitem{ti\_weight}{The 3 numbers output are 1) the logarithm of the weighting factor -beta\_P delta H,
                      2) the square of the logarithm, and 3) the weighting factor}{size: 3; }\ipiitem{time}{The elapsed simulation time.}{dimension: time; }\ipiitem{virial\_cv}{The quantum estimator for the virial stress tensor of the physical system.
                      Returns the 6 independent components in the form $[$xx, yy, zz, xy, xz, yz$]$.}{dimension: pressure; size: 6; }\ipiitem{virial\_md}{The virial tensor of the (extended) classical system. Returns the 6
                      independent components in the form $[$xx, yy, zz, xy, xz, yz$]$.}{dimension: pressure; size: 6; }\ipiitem{volume}{The volume of the cell box.}{dimension: volume; }